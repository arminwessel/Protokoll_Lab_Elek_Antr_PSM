\begin{figure}[ht!]
    \centering
    \begin{tikzpicture}
    \begin{axis}[
       width=\textwidth,
       height= 12cm,
	   xlabel=Felderregerstrom $I_{f}$,
	   x unit = \si{\ampere},
	   ylabel = bezogene Leerlaufspannung $u_{sL}$,
	   y unit = 1,
	   xtick={0,1,...,5,6,7,...,10}, 
	   xticklabels={0,1,...,5,6,7,...,10}, 
	   extra x ticks = {4.5},
	   extra x tick labels = {$I_{\mathrm{fL}}$},
	   grid=major,
        xmin=0,
        xmax=8,
		ymin=0,
		ymax=1.25,
		legend style={
            at={(0.1,0.9)},
            anchor=west}
		]
		\addplot[smooth,blue,line width=0.5mm] table[x=I_f, y=u_sL, header=has colnames,col sep=comma] {\currfiledir leerlauf_neu.csv};
		\addlegendentry{gemessen}
		\addplot[smooth,red,line width=0.5mm] table[x=I_f_corr, y=u_sL_corr, header=has colnames,col sep=comma] {\currfiledir leerlauf_neu.csv};
		\addlegendentry{korrigiert}
% 		\addplot[smooth,red,line width=0.5mm,domain=0:10, samples=100]{1.4*(1-exp(-x/3.6))};
% 		\addlegendentry{berechnet}
	\end{axis}
	\end{tikzpicture}
	\caption{Gemessene und korrigierte Leerlaufkennlinie der Synchronmaschine}
	\label{abb:SM_Leerlaufkennlinie}
\end{figure}
