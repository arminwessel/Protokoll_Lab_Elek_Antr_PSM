\begin{figure}[h!]
    \centering
    \begin{tikzpicture}
    \begin{axis}[
       width=\textwidth,
       height= 12cm,
	   xlabel=Felderregerstrom $I_{f}$,
	   x unit = \si{\ampere},
	   ylabel = bezogene Leerlaufspannung $u_{sL}$,
	   y unit = 1,
	   xtick={0,1,...,3,4,5,...,10}, 
	   xticklabels={0,1,...,3,{},5,...,10}, 
	   extra x ticks = {4.4},
	   extra x tick labels = {$I_{\mathrm{fK}}$},
	   grid=major,
        xmin=0,
        xmax=10,
		ymin=0,
		ymax=1.2
		]
		%\addplot[smooth,blue,line width=0.5mm] table[x=I_f, y=u_sL, header=has colnames,col sep=comma] {./4/leerlauf.csv};
		%\label{kurzschluss1}
		\addplot[smooth,red,line width=0.5mm,domain=0:10, samples=100]{1.4*(1-exp(-x/3.6))};
		%\label{kurzschluss2}
	\end{axis}
	\begin{axis}[
       width=\textwidth,
       height= 12cm,
       axis x line = none,
       yticklabel pos = right,
	   ylabel = bezogener Kurzschlussstrom $i_{\mathrm{sK}}$,
	   y unit = \si{\ampere},
        xmin=0,
        xmax=10,
		ymin=0,
		ymax=1.2,
		legend style={
            at={(0.1,0.9)},
            anchor=west}
		]
		%\addlegendimage{/pgfplots/refstyle=kurzschluss1}\addlegendentry{$u_{sL}$ Gemessen}
		\addlegendimage{/pgfplots/refstyle=kurzschluss2}\addlegendentry{$u_{sL}$ Berechnet}
		\addplot[smooth,blue,dashed,line width=0.5mm] table[x=I_f, y=i_s, header=has colnames,col sep=comma] {\currfiledir kurzschluss.csv};
		\addlegendentry{$I_{sK}$}
		\end{axis}
	\end{tikzpicture}
	\caption{Gemessene und korrigierte ($\Delta$) Leerlaufkennlinie und Kurzschluss}
	\label{abb:SM_Kurzschluss}
\end{figure}
