\clearpage 

\subsection{Feldschwächbetrieb} 
%Die PSM ist schlecht feldschwächbar! 
%:D
Ziel des Feldschächbetriebs ist es die Maschine über der Nenndrehzahl $n_N$ zu betreiben, ohne dabei die Nennstatorspannuns $U_{S,N}$ zu überschreiten. Wird eine negative $d$-Komponente des Stromes eingeprägt, bewirkt das einen Fluss der dem Fluss des Permanentmagneten entgegengesetzt ist. Durch den geringeren Flussbetrag sinkt die induzierte Spannung, und damit die Außenleiterspannung. Die $d$-Komponente des Stromes leistet keinen Beitrag zum Drehmoment. Für den Versuch wurde bei Nenndrehzahl ein bezogener Strom $i_d$ in negative Richtung eingeprägt. Dabei wurde die Außenleiterspannung $U_{AL}$ gemessen. Die Messwerte sind in Tabelle \ref{tab:PSM_feldschwaech} dargestellt.

\begin{table}[!ht]
\centering% Tabelle zu Kurzschluss
    \begin{tabular}{|l|c|}
    \hline
    $i_d [1]$ & $U_{AL} [V]$ \\ \hline
    0         & 107          \\ \hline
    -0.1      & 103          \\ \hline
    -0.2      & 99           \\ \hline
    -0.3      & 96           \\ \hline
    -0.4      & 92           \\ \hline
    -0.5      & 88           \\ \hline
    \end{tabular}
    \caption{Messwerte zum Feldschwächbetrieb}
    \label{tab:PSM_feldschwaech}
\end{table}
\noindent In Abbildung \ref{fig:feldschwaech} sind links die gemessenen Punkte dargestellt. Rechts ist die aus den Messdaten extrapolierte Gerade eingezeichnet, zusammen mit dem Wert
\begin{equation*}
    i_{d,0} = -2.855,
\end{equation*}
bei der die extrapolierte Außenleiterspannung Null wird. Man kann daraus auf die bezogene Induktivität der Maschine in d-Richtung schließen:
\begin{equation*}
    x_d = \frac{1}{|i_{d,0}|} \approx 0.35,
\end{equation*}
was dem typischen Wert einer PSM von $x_d \approx \frac{1}{3}$ entspricht.
\input{\currfiledir aussenleiterspannung.tex}