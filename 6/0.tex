\subsection{Synchronisation an das Spannungsnetz}
Im nächsten Schritt muss die Synchronmaschine mit dem Netz synchronisiert werden. Dabei müssen folgende Bedingungen erfüllt werden für ein stoßfreies Zuschalten:
\begin{itemize}
    \item gleiche Spannungsamplituden
    \item gleiche Frequenz
    \item gleiche Phasenlage
    \item gleiche Phasenfolge
\end{itemize}
Die sogenannte Hell-Dunkel-Schaltung wurde für den Abgleich genutzt. Die zugrunde liegende Schaltung kann dem Laborskript entnommen werden. Der Aufbau im Labor hat jeweils 2 redundante Lampen, damit es unter keinem Umständen zu einer falschen Zuschaltung kommt, weil eine Lampe ausgefallen ist.\\
Die Spannungsamplitude wurde einfachheitshalber über ein Voltmeter überprüft. Falls die Spannungen nicht übereinstimmen, muss der Erregerstrom der Synchronmaschine entsprechend nachjustiert werden.\\
Der Frequenzabgleich ist dann gegeben, wenn die Helligkeit der Lampen sih nicht ändert. Bei einer Frequenzdifferenz  stellt sich eine fortlaufende Spannungsänderung ein, die über die Lampen visualisiert wird. Um die Frequenz anzupassen muss die Drehzahl der Maschine über die gekoppelte GSM geändert werden.\\
Die Phasenfolge ist dann richtig, wenn die "richtigen" Lampen hell bzw. dunkel sind. Bei vertauschen der Phasenlage würden die falschen Lampen hell/dunkel sein. 
Die Phasenlage sind dann gleich, wenn die Dunkellamoe dunkel ist und die beiden anderen Lampen gleich hell sind. Ist dies nicht der Fall muss die Drehzahl kurzzeitig geändert werden, damit sich die Phasen verschieben können.\\
Sind alle Bedingungen erfüllt kann die Maschine über einen Schalter mit dem Netz verbunden werden.
