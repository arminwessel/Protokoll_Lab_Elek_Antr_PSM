\begin{figure}[h!]
    \centering
    \begin{tikzpicture}
    \begin{axis}[
       width=\textwidth,
       height= 10cm,
	   xlabel=Zeit $t$,
	   x unit = \si{\milli\second},
	   grid=both,
	   minor tick num=3,
        xmin=0,
        xmax=500,
		ymin=-10,
		ymax=10,
		legend style={
            at={(0.4,0.15)},
            anchor=west}]
		\addplot[smooth,green,line width=0.3mm] table[x=t, y=omega, header=has colnames,col sep=comma,x expr=\thisrowno{0}*1000+250] {\currfiledir umkehr_bldc_dq.csv};
		\addlegendentry{$\omega_m$}
		\addplot[smooth,blue,line width=0.3mm] table[x=t, y=id, header=has colnames,col sep=comma,x expr=\thisrowno{0}*1000+250] {\currfiledir umkehr_bldc_dq.csv};
		\addlegendentry{$i_d$}
		\addplot[smooth,red,line width=0.3mm] table[x=t, y=iq, header=has colnames,col sep=comma,x expr=\thisrowno{0}*1000+250] {\currfiledir umkehr_bldc_dq.csv};
		\addlegendentry{$i_q$}
		\addplot[smooth,orange,line width=0.3mm] table[x=t, y=gamma, header=has colnames,col sep=comma,x expr=\thisrowno{0}*1000+250] {\currfiledir umkehr_bldc_dq.csv};
		\addlegendentry{$\gamma_m$}
		\end{axis}
	\end{tikzpicture}
    \caption{Zeitverlauf der Ströme, des Winkels und der Drehzahl bei einem Drehzahlsprung im BLDC-Betrieb im d,q-KOS.}
    \label{fig:umkehr_bldc_dq}
\end{figure}
