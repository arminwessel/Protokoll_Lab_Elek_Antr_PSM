\begin{figure}[h!]
	\centering{
		\begin{tikzpicture}
		\pgfplotsset{set layers}
		% Variablen festlegen
		\pgfmathsetmacro{\Un}{1}
		\pgfmathsetmacro{\In}{1}
		\pgfmathsetmacro{\IfIV}{10.4}
		\pgfmathsetmacro{\IfK}{4.4}
		\pgfmathsetmacro{\IfL}{4.5}
		\pgfmathsetmacro{\tau}{3.6}
		\pgfmathsetmacro{\kLL}{0.388} % Steigung der Leerlaufkennlinie
		\pgfmathsetmacro{\kcnull}{\In / \IfK *\Un/\kLL}
		\pgfmathsetmacro{\kc}{1}
		\begin{axis}[
		width=0.9*\textwidth,
		height=12cm,
		xlabel=$I_{\mathrm{f}}$,
		x unit=\si{\ampere},
		ylabel={$u_S$,$i_S$},
		y unit=1,
		grid = major,
		ymin=0,
		ymax=1.4,
		xmin=0,
		xmax=12,
		xtick={0,2,4,6,8,10,12}, 
		xticklabels={0,2,{},6,8,{},12},
		ytick={0,0.2,0.4,0.6,0.8,1,1.2,1.4},
		yticklabels={0,$U_{\mathrm{iK}}$,\num{0.4},\num{0.6},\num{0.8},1,$U_{\mathrm{iIV}}$,\num{1.4}},
		extra x ticks={4.4,10.4}, 
		extra x tick labels={{$I_{\mathrm{fK}}, I_{\mathrm{fL}}$},$I_{\mathrm{fIV}}$},
		extra tick style={grid=none,xtick style={draw=none}},
		legend style={
            at={(0.7,0.1)},
            anchor=west}
		]
		
		% Koordinaten festlegen
		\coordinate (IV) at (\IfIV, \Un);
		\coordinate (IVsub) at ([shift=(IV)] -\IfK,0);
		
		% Induktiver Volllastpunkt
		\draw[dashed] (\IfIV,0) -- (IV);
		
		% I_{fK} von I_{IV} subtrahieren
		\draw[thick, blue] (IV) -- (IVsub);
		\draw [fill,blue] (IV) ellipse [x radius=0.08cm, y radius=0.08cm];
		\draw [fill,blue] (IVsub) ellipse [x radius=0.08cm, y radius=0.08cm];
		
		% Leerlaufkennlinie
		\addplot[smooth,red,line width=0.5mm,domain=0:10, samples=100, name path=leerlauf]{1.4*(1-exp(-x/\tau))};		
		\addlegendentry{$u_{\mathrm{sL}}$}
		
		% Anfangssteigung der Leerlaufkennlinie
		\addplot[domain=0:\Un/\kLL, green,dashed] {x*\kLL};
		\addlegendentry{$\Diff{U_{\mathrm{sL}}}{I_{\mathrm{f}}}\vert_{I_f=0}$}
		
		%Kurzschlusskennlinie
		\addplot[domain=0:7, red] {\In / \IfK *x};
		\addlegendentry{$i_{\mathrm{S}}$}
		
		% Parallelverschieben
		\addplot[domain=0:15,gray, thin, name path=parallel] {\kLL *x - (\kLL * (\IfIV - \IfK) - \In)};
		\path[name intersections={of=leerlauf and parallel,name=UiIV}];
		\coordinate (UiIV) at (UiIV-1);
		
		\draw[blue,thick] (IVsub) -- (UiIV);
		
		% Dreieck fertig zeichnen
		\draw[blue,thick] (UiIV) -- (IV);
		
		% Potierreaktanz
		\draw[dashed] (UiIV |- 0,\Un) -- node[right]{$x_p$} (UiIV);
		
		% innere Spannung im induktiven Vollastpunkt
		\draw[dashed] (UiIV -| 0,0) -- (UiIV);
		
		% oberer Punkt im Dreieck
		\draw [fill,blue] (UiIV) ellipse [x radius=0.08cm, y radius=0.08cm];
		
		% innere Spannung im Kurzschlusspunkt
		\coordinate (UiK) at ([shift=(UiIV)]  \IfK - \IfIV,-\Un);
		\draw[thick, blue] (0,0) -- (UiK);
		\draw[thick, blue] (UiK) -- (\IfK,0);
		\draw[dashed] ( UiK -| 0,0 ) -- (UiK);
		
		% auf Felderregerseite umgerechnete Ankernennstrom
		\draw [decorate,decoration={brace,amplitude=5pt,aspect=0.67},xshift=0pt,yshift=-3pt]
		(IV) -- (UiIV |- 0,\Un) node [black,pos=0.67,yshift=-13pt] 
		{$I_{\mathrm{sN}}''$};
		
		% kc 
		\draw[thick] (\IfL,0) -- node[right]{$k_c$} (\IfL,\In / \IfK *\IfL);
		%\draw[dashed] (\IfL,\In / \IfK *\IfL) -- (\IfL,\In );
		
		% kc0
		\draw[thick] (axis cs: \Un/\kLL,0) -- node[left]{$k_{c0}$} (axis cs: \Un/\kLL,\kcnull);
		\draw[dashed] (axis cs: \Un/\kLL,\kcnull) -- (axis cs: \Un/\kLL,\In);
		
		% IfK
		\draw[dashed] (axis cs:\IfK,0) -- (axis cs:\IfK,\In);
		\end{axis}
		\end{tikzpicture}
	}
	\caption{Konstruktion nach Fischer-Hinnen für die Bestimmung der Potierreaktanz.}
    \label{abb:Fischer_Hinnen}
\end{figure}
