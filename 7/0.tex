\subsection{Bestimmung der Potierreaktanz}
\label{subsec:potierreaktanz}
Nachdem die Maschine erfolgreich ans Netz gekoppelt wurde, soll die Potierreaktanz bestimmt werden. Dazu wird die graphische Methode nach Fischer-Hinnen genutzt.\\
Zusätzlich zur Leerlauf- und Kurzschlusskennlinie wird noch der induktive Volllastpunkt $IV$ benötigt, damit alle relevanten Maschinendaten bestimmt werden können. Der Induktive Volllastpunkt ist dadurch gekennzeichnet, dass die Maschine nur Blindleistung abgibt und keine Wirkleistung. Die Maschine wirkt also kapazitiv aus Sicht des Netzes.\\
Um die Synchronmaschine in den induktiven Volllastpunkt zu bringen, wurde der Erregerstrom erhöht bis sich Nennstrom und Nennspannung einstellen. Die Verluste der Synchronmaschine werden von der GSM gedeckt, indem das Drehmoment (bzw. der Erregerstrom der GSM) so gewählt wird, dass keine Wirkleistung aufgenommen wird. Der induktive Volllastpunkt ist in Abbildung \ref{abb:Fischer_Hinnen} dargestellt und liegt bei ca. $I_fIV = \SI{10.4}{\ampere}$.\\
Für die Bestimmung der Potierreaktanz muss nun der Strom $I_{fK}= \SI{4.4}{\ampere}$ vom Strom $I_{fIV}$ subtrahiert werden und ins Diagramm eingetragen werden. Durch Parallelverschiebung der Anfangssteigung der Leerlaufkennlinie durch eben erhaltenen Punkt kann mit dem Schnittpunkt der Leerlaufkennlinie die innere Spannung im induktiven Volllastpunkt $u_{iIV} = 1.163$ abgelsen werden. Die Höhe des entstanden Dreiecks entspricht der Potierreaktanz $x_p = 0.163$. Die Teillänge $I''_{sN}= \SI{4}{\ampere}$ wird für den Umrechnungsfaktor $\gamma$ gebraucht.
\begin{equation*}
    \gamma = \frac{I''_{sN}}{I_{sN}} = 0.069
\end{equation*}
Darüber hinaus lassen sich die gesättigte ($k_ {c}$) und ungesättigte ($k_ {c0}$) Leerlauf-Kurzschluss-Verhältnisse graphisch ablesen.
\begin{equation*}
    k_ {c0} = \frac{I_{f0}}{I_{fK}} = \frac{\SI{2.6}{\ampere}}{\SI{4.4}{\ampere}} 0.59, \quad k_ {c} = \frac{I'_{f0}}{I_{fK}} = \frac{\SI{4.53}{\ampere}}{\SI{4.4}{\ampere}} 1.03
\end{equation*}
Aus diesen können die gesättigte und die ungesättigte bezogene synchrone Längsreaktanz abgeleitet werden.
\begin{equation*}
    x_d = \frac{1}{k_c} = 1.69, \quad x_{d0} = \frac{1}{k_{c0}} = 0.97
\end{equation*}
Die charakteristischen Ströme können durch die Beziehung
\begin{equation*}
    i_f = \frac{I'_f}{I_{sN}} = \frac{I_f}{\gamma I_{sN}}
\end{equation*}
auf die Statorseite umgerechnet werden. Die aus dem Diagramm entnommenen Werte sind in der Tablle \ref{tab:Fischer_Hinnen_abgelesene_Werte} zusammengefasst, die daraus berechneten Werte können der Tabelle \ref{tab:Fischer_Hinnen_berechnete_Werte} entnommen werden.

% Tabelle der abgelesenen Werte
\begin{table}[!ht]
\centering
\begin{tabular}{|c|c|}
\hline
            & abgelesene Werte      \\ \hline
$I_{fK}$    &  \SI{4.4}{\ampere}    \\ \hline
$I_{fL}$    & \SI{4.53}{\ampere}    \\ \hline
$I_{fIV}$   & \SI{10.4}{\ampere}    \\ \hline
$I''_{sN}$  & \SI{4}{\ampere}       \\ \hline
$x_p$       & $0.163$               \\ \hline
$u_{iIV}$   & $1.163$               \\ \hline
$u_{iK}$    & $0.163$               \\ \hline
$k_c$       & $0.59$                \\ \hline
$k_{c0}$    & $1.03$                \\ \hline
\end{tabular}
\caption{Die abgelesenen Werte aus LL- und KS-Versuch nach Fischer-Hinnen}
\label{tab:Fischer_Hinnen_abgelesene_Werte}
\end{table}

% Tabelle der berechneten Werte
\begin{table}[!ht]
\centering
\begin{tabular}{|c|c|}
\hline
            & berechnete Werte  \\ \hline
$i_{fK}$    &  $1.105$          \\ \hline
$i_{fL}$    & $1.137$           \\ \hline
$i_{fIV}$   & $2.612$           \\ \hline
$x_d$       & $1.69$            \\ \hline
$x_{d0}$   & $0.97$            \\ \hline
\end{tabular}
\caption{berechnete charakteristische Größen}
\label{tab:Fischer_Hinnen_berechnete_Werte}
\end{table}
Die synchrone Hauptfeldreaktanz $x_{dh}$ wurde abschließend als Funktion des bezogenen Magnetisierungsstroms $i_{md}$ bestimmt. Es gilt allgemein
\begin{equation*}
    \label{eq:Hauptfeldreaktanz}
    x_{dh} \big|_{i_{md}} = \frac{u_{iq} \big|_{i_{md}}}{i_{md}}
\end{equation*}
Der bezogene Magnetisierungsstrom ist durch 
\begin{equation*}
    i_{md} = i_f + i_{sd}
\end{equation*}
gegeben. Für den Leerlauffall gilt $i_s = 0 \rightarrow i_{sd} = 0$ und somit vereinfacht sich Gleichung \ref{eq:Hauptfeldreaktanz} zu:
\begin{equation*}
        x_{dh} \big|_{i_{f}} = \frac{u_{s} \big|_{i_{f}}}{i_{f}}
\end{equation*}
Der Verlauf der Hauptfeldreaktanz $x_{dh}$ ist im linearen Bereich der Leerlaufkennlinie konstant. Durch die Sättigung des Eisens kommt es in der Leerlaufkennlinie zu einem Abflachen, wodurch die Hauptfeldreaktanz ebenfalls sinkt. Im Bereich hoher Sättigung ist die Leerlaufkennlinie wieder fast linear und die Hauptfeldreaktanz wird wieder näherungsweise konstant verlaufen. Die im Punkt \ref{subsec:leerlauf} dargestellte Kennlinie 


In Abbildung \ref{abb:SM_hauptfeldreaktanz} kann die berechnete Kennlinie betrachtet werden. Die dazugehörigen Werte sind der Tabelle \ref{tab:Hauptfeldreaktanz} zu entnehmen.



% Tabelle für die Hauptfeldreaktanz
\begin{table}[!ht]
\centering
\begin{tabular}{|c|c|c|}
\hline
$u_s$    & $i_f$    & $x_{dh}$  \\ \hline
0.314    & 0.25     & 1.254     \\ \hline
0.561    & 0.5      & 1.123     \\ \hline
0.776    & 0.75     & 1.035     \\ \hline
0.938    & 1        & 0.938     \\ \hline
1.058    & 1.25     & 0.846     \\ \hline
1.136    & 1.5      & 0.758     \\ \hline
1.201    & 1.75     & 0.677     \\ \hline
1.24    & 2        & 0.620   \\ \hline
\end{tabular}
\caption{Werte für die Hauptfeldreaktanz}
\label{tab:Hauptfeldreaktanz}

\end{table}

\input{\currfiledir hauptfeldreaktanz.tex}
\input{\currfiledir fischer-hinnen.tex}