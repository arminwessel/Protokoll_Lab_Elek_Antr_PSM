\begin{figure}[h!]
    \centering
    \begin{tikzpicture}
    \begin{axis}[
       width=\textwidth,
       height= 10cm,
	   xlabel=Zeit $t$,
	   x unit = \si{\milli\second},
	   ylabel = Phasenströme,
	   %y unit = \si{\ampere},
	   grid=major,
        xmin=0,
        xmax=400,
		ymin=-2.5,
		ymax=2.5,
		legend style={
            at={(0.8,0.85)},
            anchor=west}]
		\addplot[smooth,orange,line width=0.3mm] table[x=t, y=i1, header=has colnames,col sep=comma] {\currfiledir Phasenstroeme.csv};
		\addlegendentry{$i_1$}
		\addplot[smooth,blue,line width=0.3mm] table[x=t, y=i2, header=has colnames,col sep=comma] {\currfiledir Phasenstroeme.csv};
		\addlegendentry{$i_2$}
		\addplot[smooth,red,line width=0.3mm] table[x=t, y=i3, header=has colnames,col sep=comma] {\currfiledir Phasenstroeme.csv};
		\addlegendentry{$i_3$}
		\addplot[smooth,green,line width=0.3mm] table[x=t, y=sum, header=has colnames,col sep=comma] {\currfiledir Phasenstroeme.csv};
		\addlegendentry{$i_1$+$i_2$+$i_3$}
		\end{axis}
	\end{tikzpicture}
    \caption{Zeitverlauf der Phasenströme beim Hochlauf im feldorientierten Betrieb.}
    \label{fig:phasenstroeme_feldorientiert}
\end{figure}
