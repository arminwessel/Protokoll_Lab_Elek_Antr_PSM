\begin{figure}[h!]
    \centering
    \begin{tikzpicture}
    \begin{axis}[
       width=\textwidth,
       height= 10cm,
	   xlabel=Zeit $t$,
	   x unit = \si{\milli \second},
	   grid=major,
        xmin=0,
        xmax=500,
		ymin=-10,
		ymax=10,
		legend style={
            at={(0.29,0.15)},
            anchor=west}]
		\addplot[smooth,blue,line width=0.3mm] table[x=second, y=CH1, header=has colnames,col sep=comma,x expr=\thisrowno{0}*1000+200] {\currfiledir drehzahlsprung.csv};
		\addlegendentry{$i_{\alpha}$}
		\addplot[smooth,red,line width=0.3mm] table[x=second, y=CH2, header=has colnames,col sep=comma,x expr=\thisrowno{0}*1000+200] {\currfiledir drehzahlsprung.csv};
		\addlegendentry{$i_{\beta}$}
		\addplot[smooth,green,line width=0.3mm] table[x=second, y=CH3, header=has colnames,col sep=comma,x expr=\thisrowno{0}*1000+200] {\currfiledir drehzahlsprung.csv};
		\addlegendentry{$\omega_m$}
		\addplot[smooth,orange,line width=0.3mm] table[x=second, y=CH4, header=has colnames,col sep=comma,x expr=\thisrowno{0}*1000+200] {\currfiledir drehzahlsprung.csv};
		\addlegendentry{$\gamma_m$}
		\end{axis}
	\end{tikzpicture}
    \caption{Zeitverlauf der Ströme, des Winkels und der Drehzahl beim Drehzahlsprung im Sinus-Betrieb.}
    \label{fig:sprung_sinus}
\end{figure}
