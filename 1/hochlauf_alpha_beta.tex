\begin{figure}[h!]
    \centering
    \begin{tikzpicture}
    \begin{axis}[
       width=\textwidth,
       height= 10cm,
	   xlabel=Zeit $t$,
	   x unit = \si{\milli\second},
	   grid=major,
        xmin=0,
        xmax=450,
		ymin=-10,
		ymax=10,
		legend style={
            at={(0.05,0.85)},
            anchor=west}]
		\addplot[smooth,red,line width=0.3mm] table[y=1,x expr=\thisrow{x-axis}*1000+117,header=has colnames,col sep=comma] {\currfiledir hochlauf_alpha_beta.csv};
		\addlegendentry{$i_{\alpha}$}
		\addplot[smooth,blue,line width=0.3mm] table[y=2,x expr=\thisrow{x-axis}*1000+117, header=has colnames,col sep=comma] {\currfiledir hochlauf_alpha_beta.csv};
		\addlegendentry{$i_{\beta}$}
		\addplot[smooth,green,line width=0.3mm] table[y=3,x expr=\thisrow{x-axis}*1000+117, header=has colnames,col sep=comma] {\currfiledir hochlauf_alpha_beta.csv};
		\addlegendentry{$\omega_m$}
		\addplot[smooth,orange,line width=0.3mm] table[y=4,x expr=\thisrow{x-axis}*1000+117, header=has colnames,col sep=comma] {\currfiledir hochlauf_alpha_beta.csv};
    	\addlegendentry{$\gamma_m$}
		\end{axis}
	\end{tikzpicture}
    \caption{Zeitverlauf der Ströme, des Winkels und der Drehzahl beim Einschaltvorgang und halbem Nennmoment im $\alpha,\beta$-KOS.}
    \label{fig:hochlauf_alpha_beta}
\end{figure}
